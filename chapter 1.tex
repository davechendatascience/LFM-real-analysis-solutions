\documentclass[12pt]{article}
\usepackage{graphicx} % Required for inserting images
\usepackage{bm}
\usepackage{amsmath,amssymb,amsthm,enumitem}
\usepackage{mathtools}
\usepackage{listings}
\usepackage{mathrsfs}
\usepackage{tikz}

\title{Real Analysis - A Long Form Mathematics Textbook Chapter 1}
\author{Yen-Ting Chen}
\date{May 2024}

\tikzset{
	% define the bar graph element
	bar/.pic={
		\fill (-.1,0) rectangle (.1,#1) (0,#1) node[above,scale=1/2]{$#1$};
	}
}

\begin{document}
	\maketitle
	\section*{1.5 The Completness Axiom}
		The set R has two binary operations, addition(+) and multiplication (*), and is the unique set satisfying the following axioms:
		\begin{itemize}
			\item Axiom 1: (Commutative Law), If a, b $\in$ R, then $a+b = b+a$ and $a*b = b*a$
			\item Axiom 2: (Distributive Law), If a,b,c $\in$ R, then $a*(b+c) = a*b + a*c$
			\item Axiom 3: (Associative Law), If a,b,c $\in$ R, then $(a+b)+c = a+(b+c)$ and $(a*b)*c = a*(b*c)$
			\item Axiom 4: (Identity Law). There are special elements 0,1 $\in$ R, where $a+0 = a$ and $a*1=a$ for all a $\in$ R
			\item Axiom 5: (Inverse Law). For each a $\in$ R, there is an element -$a \in R$ such that $a + (-a) = 0$. If a $\ne$ 0, then there is also an element $a^{-1} \in R$ such that $a*a^{-1} = 1$
			\item Axiom 6: (Order Axiom). There is nonempty subset $P \subseteq R$, called the positive elements, such that
				\begin{itemize}
					\item If a,b $\in$ P, then a + b $\in$ P and a*b $\in$ P
					\item If a $\in$ R and a $\ne$ 0, then either a $\in$ P or -a $\in$ P, but not both.
				\end{itemize}
			\item Axiom 7: (Completeness Axiom). Given any nonempty A $\subseteq$ R where A is bounded above, A has a least upper bound. In other words, $sup(A) \in R$ for every such A.
		\end{itemize}
		
		Theorem 1.24 (Suprema analytically). Let A $\subseteq$ R. Then $sup(A) = \alpha$ if and only if
		\begin{itemize}
			\item $\alpha$ is an upper bound of A, and
			\item Given any $\epsilon > 0, \alpha - \epsilon$ is not an upper bound of A. That is, there is some $x \in A$ for which $x > \alpha - \epsilon$
		\end{itemize}
		Likewise, $inf(A) = \beta$ iff
		\begin{itemize}
			\item $\beta$ is a lower bound of A, and
			\item Given any $\epsilon > 0$, $\beta + \epsilon$ is not a lower bound of A. That is there is some $x \in A$ for which $x < \beta + epsilon$
		\end{itemize}
	\section*{1.7 The Archimedean Principle}
		Lemma 1.26 (The Archimedean Property). If a and b are real numbers with a > 0, then there exists a natural number n such that $na > b$. In particular, for any $\epsilon > 0$ there exists $n \in N$ such that $1/n < \epsilon$.
		
		Principle 1.34 (Well-ordering principle). Every non-empty subset of natural numbers contains a smallest element.
	\section*{Exercises}
		\subsection*{Exercise 1.1}
			Explain the error in the following proof that 2=1. Let x = y. Then
			\begin{equation}
				\begin{split}
					x^2 = xy \\
					x^2 - y^2 = xy - y^2 \\
					(x+y)(x-y) = y(x-y) \\
					x+y = y \\
					2y = y\\
					2 = 1
				\end{split}
			\end{equation}
			
			answer: by dividing by (x-y), the proof is essentially dividing by zero, which is mathematicall undefined; thus this invalidates the entire proof.
		\subsection*{Exercise 1.2}
			Which of the following statements are true? Give a short explanation for each of your answers.
			\begin{itemize}
				\item For ever $n \in N$, there is $m \in N$ such that $m > n$.
				\item For every $m \in N$, there is an $n \in N$ such that $m > n$.
				\item There is an $m \in N$ such that for every $n \in N$, $m \ge n$.
				\item There is an $n \in N$ such that for every $m \in N$, $m \ge n$.
				\item There is an $n \in R$ such that for every $m \in R$, $m \ge n$.
				\item For every pair $x<y$ of integers, there is an integer z such that $x < z < y$.
				\item For every pair $x < y$ of real numbers, there is a real number z such that $x < z < y$.
			\end{itemize}
			answer:
			\begin{itemize}
				\item True. For any natural number 
				n, choosing $m=n+1$ satisfies $m > n$. Natural numbers are unbounded above.
				\item False. If m=1, there is no $n \in N$ with $n < 1$.
				\item False. There is no largest natural number; N is infinite.
				\item True. Let n=1. For all $m \in N$, $m \ge 1$.
				\item False. Real numbers extend to $-\infty$; no universal lower bound $n \in R$.
				\item False. If x and y are consecutive integers (e.g., $x=2,y=3$), no integer z exists between them.
				\item True. For real numbers, 
				$z = (x+y)/2$ always satisfies $x<z<y$.
			\end{itemize}
		\subsection*{Exercise 1.3}
			If A and B are two boxes (possibly with things inside), describe the following in terms of boxes:
			\begin{itemize}
				\item $A \ B$
				\item $P(A)$
				\item $|A|$
			\end{itemize}
			answers:
			\begin{itemize}
				\item Imagine looking inside box $A$ and taking out any item that is also found in box $B$. What remains in $A$ are only those things that are not in $B$.
				\item Think of every possible way you could select items from box $A$ (including selecting none, or all). Each possible selection is itself a (possibly empty) box. The collection of all these possible boxes is $P(A)$.
				\item Open box $A$ and count how many items are inside. That count is $|A|$.
			\end{itemize}
		\subsection*{Exercise 1.4}
			If $A_1, A_2, A_3, ..., A_n$ are all boxes (possibly with things inside), describe the following terms of boxes:
			\begin{itemize}
				\item $\bigcup_{i=1}^{n}A_i$
				\item $\bigcap_{i=1}^{n}A_i$
			\end{itemize}
			answers:
			\begin{itemize}
				\item If an item exists in at least one box, it appears once in the union box.
				\item If even one box lacks an item, it is excluded from the intersection.
			\end{itemize}
		\subsection*{Exercise 1.5}
			Prove that each of the following holds for any sets A and B.
			\begin{itemize}
				\item $A \cup B = A$ iff $B \subseteq A$
				\item $A \cap B$ iff $A \subseteq B$
				\item $A \ B = A$ iff $A \cap B$ = $\emptyset$
				\item $A \ B = \emptyset$ iff $A \subseteq B$
			\end{itemize}
			answers:
			\begin{itemize}
				\item (forward)If $A \cup B = A$, every element of B must already be in A. Thus $B \subseteq A$. (reverse) If $B \subseteq A$, combining A and B adds no new elements, so $A \cup B = A$
				\item (forward) if $A \cap B = A$, all elements of A are in B, so $A \subseteq B$. (reverse) if $A \subseteq B$, the intersection $A \cap B$ contains exactly A.
				\item (forward) if $A \setminus B = A$, no elements of A are in B, so $A \cap B$ = $\emptyset$. (reverse) if $A \cap B$ = $\emptyset$, removing B from A leaves A unchanged.
				\item (forward) if $A \setminus B = \emptyset$, all elements of A are in B, so $A \subseteq B$. (reverse) if $A \subseteq B$, removing B from A removes all elements, leaving $\emptyset$
			\end{itemize}
		\subsection*{Exercise 1.6}
			Suppose f:$X \rightarrow Y$ and $A \subseteq X$ and $B \subseteq Y$.
			\begin{itemize}
				\item Prove that $f(f^{-1}(B)) \subseteq B$
				\item Give an example where $f(f^{-1}(B)) \ne B$
				\item Prove that $A \subseteq f^{-1}(f(A)))$
				\item Give an example where $A \ne f^{-1}(f(A)))$
			\end{itemize}
			answers:
			\begin{itemize}
				\item Let $y \in f(f^{-1}(B))$. By definition, there exists $x \in f^{-1}(B)$ such that $f(x) = y$. Since $x \in f^{-1}(B)$, $f(x) \in B$ by the definition of preimage. Thus $y = f(x) \in B$. Therefore, $f(f^{-1}(B)) \subseteq B$
				\item Let f: $\{1,2\} \rightarrow \{a,b,c\}$ with $f(1) = a$ and $f(2) = b$. Take $B = \{a,b,c\}. f^{-1}(B) = \{1,2\}. f(f^{-1}(B)) = f(\{1,2\}) = \{a,b\} \ne B$
				\item Let $x \in A$. Then $f(x) \in f(X)$. By definition of preimage, $x \in f^{-1}(f(A))$. Thus, $A \subseteq f^{-1}(f(A))$
				\item Let f: $\{1,2\} \rightarrow \{a\}$ with $f(1) = f(2) = a$. Take $A = \{1\}. f(A) = \{a\}$. $f^{-1}(f(A)) = f^{-1}(\{a\}) = \{1,2\} \ne A$
			\end{itemize}
		\subsection*{Exercise 1.7}
			Suppose that f:$X \rightarrow Y$ and g: $Y \rightarrow X$ are functions and that the composite g of f is the identity function id" $X \rightarrow X$. (The identity function sends every element to itself: $id(x)=x$) Show that f must be a one-to-one function and that g must be an onto function.
			
			answer:
			\begin{itemize}
				\item f is injective
				
				Assume $f(x_1) = f(x_2)$ for some $x_1, x_2 \in X$. Applying g to both sides: $g(f(x_1)) = g(f(x_2))$ since g of f = $id_X$, this simplifies to: $x_1 = x_2$. Thus, $f(x_1) = f(x_2) \rightarrow x_1 = x_2$, proving f is injective.
				\item g is surjective
				
				For any $x \in X$, let $y = f(x) \in Y$. Applying g to y: $g(y) = g(f(x)) = x$. Thus, every $x \in X$ has preimage $y = f(x) \in Y$ under g, proving g is surjective.
			\end{itemize}
		\subsection*{Exercises 1.8}
			The following are special cases of De Morgan's laws
			\begin{itemize}
				\item Prove that $(A \cap B)^c = A^c \cup B^c$
				
					(a) $(A \cap B)^c \subseteq A^c \cup B^c$
					
					Let $x \in (A \cap B)^c$. Then $x \notin A \cap B$, so $x \notin A$ or $x \notin B$. Thus, $x \in A^c$ or $x \in B^c$. Therefore $x \in A^c \cup B^c$.
					
					(b) $A^c \cup B^c \subseteq (A \cap B)^c$
					
					Let $x \in A^c \cup B^c$.
					Then $x \in A^c$ or $x \in B^c$, so $x \notin A$ or $x \notin B$. Thus, $x \notin A \cap B$, so $x \in (A \cap B)^c$ 
				\item Prove that $(A \cup B)^c = A^c \cap B^c$
					
					(a) $(A \cup B)^c \subseteq A^c \cap B^c$
					
					Let $x \in (A \cup B)^c$.
					Then $x \notin A \cup B$, so $x \notin A$ and $x \notin B$. Thus, $x \in A^c$ and $x \in B^c$, so $x \in A^c \cap B^c$.
					
					(b) $A^c \cap B^c \subseteq (A \cup B)^c$
					
					Let $x \in A^c \cap B^c$.
					Then $x \in A^c$ and $x \in B^c$, so $x \notin A and x \notin B$. Thus $x \notin A \cup B$, so $x \in (A \cup B)^c$. Therefore, $(A \cup B)^c = A^c \cap B^c$
			\end{itemize}
		\subsection*{Exercise 1.9}
			\begin{itemize}
				\item Prove that $\sqrt{3}$ is irrational
				
				Assume $\sqrt{3}$ is rational, so $\sqrt{3} = a/b$ where $a,b \in Z$ are coprime. Squaring both sides:
				$3 = \frac{a^2}{b^2} \implies 3b^2 = a^2$. This implies $a^2$ is divisible by 3, so a must also be divisable by 3 (by the fundamental theorem of arithmetic). Let a = 3k. Substituting:
				$3b^2 = (3k)^2 \implies 3b^2 = 9k^2 \implies b^2 = 3k^2$. Thus, $b^2$ is divisible by 3, so b is also divisible by 3. This contradicts a and b being coprime. Hence, $\sqrt{3}$ is irrational.
				\item What goes wrong when you try to adapt your argument from part (a) to show that $\sqrt{4}$ is irrational?
				\item In part (a) you proved that $\sqrt{3}$ to be irrational, and essentially the same proof shows that $\sqrt{5}$ is irrational. By considering their product or otherwise, prove that $\sqrt{3} - \sqrt{5}$ and $\sqrt{3} + \sqrt{5}$ are either both rational or both irrational. Deduce that they must both be irrational.
			\end{itemize}
		\subsection*{Exercise 1.10}
			Prove that the multiplicative identity in a field is unique.
			
			Let F be a field. Suppose there exist two multiplicative identities 1 and e in F. By definition of a multiplicative identity, for all $a \in F$:
			
			$1 \times a = a$ and $e \times a = a$
			
			Consider the case where a = e. Applying the identity property of 1:
			
			$1 \times e = e$
			
			Now consider the case where a = 1. Applying the identity property of e:
			
			$e \times 1 = 1$
			
			In a field, multiplication is commutative ($a \times b = b \times a$), so:
			
			$1 \times e = e \times 1$
			
			From (1),(2) and (3), we conclude:
			
			$e = 1$
			
			Thus, there cannot be two distinct multiplicative identities in F. The multiplicative identity is unique.
		\subsection*{Exercise 1.11}
			Given an ordered field F, recall that we defined the positive elements to be a nonempty subset $P \subseteq F$ that satisfies both the following conditions:
			
			(i) If $a,b \in P$, then $a + b \in P$ and $a \times b \in P$
			
			(ii) If $a \in F$ and $a \ne 0$, then either $a \in P$ or $-a \in P$, but not both.
			
			\begin{itemize}
				\item Give an example of some $P_1 \subseteq R$ that satisfies (i) but not (ii)
				
				$P_1$ = R(the entire set of real numbers)
				
				Satisfies (i): Closed under addition and multiplication
				
				Fails (ii): For any $a \ne 0$, both a and -a belong to $P_1$, violating the "exactly one" requirement.
				
				\item Give an example of some $P_2 \subseteq R$ that satisfies (ii) but not (i)
				
				$P_2 = \{x \in R| (x>0 and x \notin Z) or (x<0 and x \in Z)\}$
				
				Satisfies (ii): For every $a \ne 0$:
				if a is positive non-integer, $a \in P_2$. If a is a positive integer, -$a \in P_2$. If a is negative integer, $a \in P_2$. If a is negative non-integer, $-a \in P_2$.
				
				Fails (i): $0.5 \in P_2$ and $0.5 \in P_2$ but $0.5 + 0.5 = 1 \notin P_2$. $-1 \in P_2$ and $0.5 \in P_2$, but $-1 + 0.5 = -0.5 \notin P_2$.
			\end{itemize}
		\subsection*{Exercises 1.12}
			Assume that F is an ordered field and $a,b,c,d \in F$ with $a < b$ and $c < d$.
			\begin{itemize}
				\item Show that $a+c < b+d$
					Step 1: use the additive property of inequalities in ordered fields: If $a < b$, then $a+c<b+c$. If $c<d$, then $b+c < b+d$.
					
					Step 2: By transitivity of $<$:
					$a+c < b+c < b+d \implies a+c<b+d$
				
				\item Prove that it is not necessarily true that $ac < bd$.
				
					Counter example:
					
					Let F = R, and choose:
					a = -2, b = -1, c = -3, d = -2.
					\begin{itemize}
						\item a < b(since -2 < -1) and c < d (since -3 < -2)
						\item Compute ac = (-2)(-3) = 6 and bd = (-1)(-2) = 2
						\item $6 \nless 2$, so $ac < bd$ fails.
					\end{itemize}
			\end{itemize}
		\subsection*{Exercise 1.13}
			Let a,b and $\epsilon$ be elements of an ordered field.
			\begin{itemize}
				\item Show that if $a < b + \epsilon$ for every $\epsilon > 0$, then $a \le b$.
				
					Suppose, for contradiction, that $a > b$. Then $a - b > 0$. Let $\epsilon$ = $a - b$, which is a positive number. Plug this into the assumption $a < b + \epsilon = a$. This says $a < a$, which is impossible. Therefore, our assumption that $a > b$ must be false. So it must be that $a \le b$.
				\item Use part (a) to show that if $|a-b| < \epsilon$ for all $\epsilon > 0$, then $a = b$
				
					Suppose, for contradiction, that $a \ne b$. Then $|a-b| > 0$. Let $\epsilon$ = $|a-b|/2$, which is still positive. By assumption $|a-b| < \epsilon = |a-b|/2$. But this says $|a-b| < |a-b|/2$, which is impossible unless $|a-b| = 0$. Therefore, our assumption that $a \ne b$ must be false. So $a = b$.
			\end{itemize}
		\subsection*{Exercise 1.14}
			Prove that the equality |ab| = |a||b| holds for all real numbers a and b.
			
				Definition of absolute value:
				For any real number  \begin{equation*}
					x: |x| = \begin{cases}
						x &\text{if $x \ge 0$}\\
						-x &\text{if $x < 0$}
					\end{cases}
				\end{equation*}
				\begin{itemize}
					\item case 1: both $a \le 0$ and $b \le 0$
					\begin{itemize}
						\item $ab \ge 0$
						\item $|ab| = ab$
						\item $|a| = a, |b| = b$
						\item $|a||b| = ab$
						\item So, $|ab| = |a||b|$
					\end{itemize}
					\item case 2: $a \ge 0, b < 0$
					\begin{itemize}
						\item $ab \le 0$
						\item $|ab| = -(ab) = -ab$
						\item $|a|=a, |b| = -b$
						\item $|a||b| = a(-b) = -ab$
						\item So, $|ab| = |a||b|$
					\end{itemize}
					\item case 3: $a < 0, b \ge 0$
						similar to case 2
					\item case 4: $a < 0, b < 0$
					\begin{itemize}
						\item $ab \ge 0$
						\item $|ab| = (-a)(-b) = ab$
						\item $|a| = -a, |b| = -b$
						\item $|a||b| = (-a)(-b) = ab$
						\item So, $|ab| = |a||b|$
					\end{itemize}
				\end{itemize}
		\subsection*{Exercise 1.15}
			For each of the following, find all numbers x which satisfy the expression.
			\begin{itemize}
				\item $|x-4| = 7$
					
					$\{-3,11\}$
				\item $|x-4| < 7$
				
					$(-3, 11)$
				\item $|x+2| < 1$
					
					$(-3,-1)$
				\item $|x-1|+|x-2| > 1$
				
					$(-\infty, 1) \cup (2, \infty)$
				\item $|x-1|+|x+1| > 1$
				
					$R$
				\item $|x-1||x+1| = 0$
				
					$\{-1, 1\}$
				\item $|x-1||x+2| = 3$
				
					$\{\frac{-1+\sqrt{21}}{2}, \frac{-1-\sqrt{21}}{2} \}$
			\end{itemize}
		\subsection*{Exercise 1.16}
			Let $max\{x,y\}$ denote the maximum of the real numbers x and y, and let $min\{x,y\}$ denote the minimum. For example, $min\{-1,4\} = -1$, and also min\{-1,-1\} = -1. Prove that
			
				$max\{x,y\} = \frac{x+y+|y-x|}{2}$ and $min\{x,y\} = \frac{x+y-|y-x|}{2}$
				
			Then find a formula for $max\{x,y,z\}$ and $min\{x,y,z\}$.
			
			\begin{itemize}
				\item Consider two cases to prove for maximum, minimum
					\begin{itemize}
						\item $y \ge x$
							
							maximum: $\frac{x+y+|y-x|}{2} = \frac{x+y+y-x}{2} = y$
							
							minimum:
							$\frac{x+y-|y-x|}{2} = \frac{x+y-y+x}{2} = x$
						\item $y < x$
						
							maximum:
							$\frac{x+y+|y-x|}{2} = \frac{x+y+x-y}{2} = x$
							
							minimum:
							$\frac{x+y-|y-x|}{2} = \frac{x+y-x+y}{2} = y$
					\end{itemize}
					Thus the formula holds for all $x,y \in R$
				\item extension to three variables
					\begin{itemize}
						\item maximum of three variables
							$max\{x,y,z\} = max(max(x,y),z) = \frac{\frac{x+y+|y-x|}{2} + z + |z - \frac{x+y+|y-x|}{2}|}{2}$
						\item minimum of three variables
							$min\{x,y,z\} = min(min(x,y),z) = \frac{\frac{x+y-|y-x|}{2} + z - |z - \frac{x+y-|y-x|}{2}|}{2}$
					\end{itemize}
			\end{itemize}
		\subsection*{Exercise 1.17}
			Prove that if $a,b \in R$ and $0<a<b$, then $a^n<b^n$ for any positive integer n.
			\begin{itemize}
				\item base case
				
				given $0 < a < b$, the inequality holds by assumption.
				
				\item inductive step:
				
				Assume $a^k < b^k$ for some integer $k \ge 1$. $a^{k+1} < ab^k$. Since a < b, then $ab^k < bb^k = b^k+1$. Thus, $a^{k+1} < b^{k+1}$.
				
				\item By induction, $a^n < b^n$ holds for all positive integers n.
			\end{itemize}
		\subsection*{Exercise 1.18}
			Prove that if $a_1, a_2, ..., a_n$ are real numbers, then: $|a_1 + a_2 + ... + a_n| \le |a_1|+|a_2|+...+|a_n|$.
				
				Use induction based on triangle inequality of two numbers to prove.
				
		\subsection*{Exercise 1.19}
			Prove that $\sum_{k=1}^{n}\frac{1}{k(k+1)} = \frac{n}{n+1}$ for every natural number n.
			
				With partial fractions, we know $\frac{1}{k(k+1)} = \frac{1}{k} - \frac{1}{k+1}$
				\begin{equation}
					\sum_{k=1}^{n}\frac{1}{k(k+1)} = (1/1-1/2) + (1/2-1/3) + ... + (1/n - 1/(n+1)) = \frac{n}{n+1}
				\end{equation}
		\subsection*{Exercise 1.20}
			Determine which natural numbers, n, have the property that $\sqrt{n}$ is irrational.
			
			A natural number n has an irrational square root if and only if it's not a perfect square.
			
		\subsection*{Exercise 1.21}
			Let $f:X \rightarrow Y$, and assume $A_1, A_2 \subseteq X$. Show that: $f(A_1 \cap A_2) \subseteq f(A_1) \cap f(A_2)$. Recall that if A is a set, then $f(A) = \{f(x):x \in A\}$.
			
			Let $y \in f(A_1 \cap A_2)$, then there exists $x \in A_1 \cap A_2$ that results in $y = f(x)$. Since $x \in A_1 \cap A_2$, $x \in A_1$ and $x \in A_2$. Therefore, $y = f(x) \in f(A_1)$ and $y = f(x) \in f(A_1)$; this gives $y \in f(A_1) \cap f(A_2)$. Thus proves $f(A_1 \cap A_2) \subseteq f(A_1) \cap f(A_2)$.
		\subsection*{Exercise 1.22}
			Give an example of a function f, and a pair of sets A and B, for which
			\begin{equation}
				f(A_1 \cap A_2) \ne f(A_1) \cap f(A_2)
			\end{equation}
			Recall that if A is a set, then $f(A) = \{f(x):x \in A\}$.
			
			Let $f:R \rightarrow R$ be $f(x) = x^2$
			$A_1 = \{1\}, A_2 = \{-1\}$.
			\begin{equation}
				\begin{split}
					f(A_1 \cap A_2) = f(\emptyset) = \emptyset \\
					f(A_1) \cap f(A_2) = \{1\} \cap \{1\} = \{1\}
				\end{split}
			\end{equation}
		\subsection*{Exercise 1.23}
			Assume that $A \subseteq B$ and both are bounded above. Prove that $sup(A) \le sup(B)$.
			
			Suppose for contradiction, that $sup(A) > sup(B)$. Then, by the definition of supremum, there exists some $a \in A$ such that $a > sup(B)$. But since $a \in B$, this contradicts the fact that $sup(B)$ is an upper bound for B.
		\subsection*{Exercise 1.24}
			Suppose $A \subseteq R$ has a maximal element - that is, there is an element $M \in A$ such that $x \le M$ for all $x \in A$. Likewise, assume $B \subseteq R$ has a minimal element m.
			\begin{itemize}
				\item Prove that $sup(A) = M$.
				\begin{itemize}
					\item Since M is the maximal element of A, by definition $\forall x \in A, x \le M$. Thus M is an upper bound of A.
					\item To show that M is the least upper bound, suppose there exists another upper bound S of A such that $S < M$. But since $M \in A$, we must have $M \le S$. This contradicts $S < M$. Therefore, no such S exists, and M is the smallest upper bound.
				\end{itemize}
				\item Prove that $inf(B) = m$.
				\begin{itemize}
					\item Since m is the minimal element of B, by definition $\forall x \in B, x \ge m$. Thus, m is a lower bound of B.
					\item To show m is the greatest lower bound, suppose there exists another lower bound t of B such that t > m. But since $m \in B$, we must have $t \le m$. This contradicts $t > m$. Therefore, no such t exists, and m is the largest lower bound.
				\end{itemize}
			\end{itemize}
		\subsection*{Exercise 1.25}
			Suppose that A is a nonempty set containing finitely many integers. Prove by induction that A has a maximal element, and that $max(A) \in A$.
			\begin{itemize}
				\item base case: if A contains exactly one integer a, then is trivially the maximal element, and $max(A) = a \in A$.
				\item inductive step: Assume every nonempty finite set of integers with k elements has a maximal element that belongs to the set. Let $A = \{a_1, a_2, ..., a_{k+1}\}$. $A' = A \setminus \{x\}$ with k elements. By the induction hypothesis, A' has a maximal element $m \in A'$. Compare x with m, if x > m, then x is the maximal element of A; if $x \le m$, then m remains the maximal element of A. This proves that the maximal element of A is an element of A. By mathematical induction, every nonempty finite set of integers has a maximal element that belongs to the set.
			\end{itemize}
		\subsection*{Exercise 1.26}
			Prove that N is complete
			
			To prove that N is complete, we first clarify the definition of completeness. Completeness means that every nonempty subset of N that is bounded above has a least upper bound in N.
			\begin{itemize}
				\item Let $S \subseteq$ N be a nonempty subset bounded above. 
				
				By definition, there exsists some $M \in N$ such that $s \le M$ for all $s \in S$
				\item S is finite. 
				
				Since $S \subseteq \{1,2,...,M\}$ and this set is finite, S must also be finite.
				\item Every finite nonempty subset of N has a maximum.
				
				By the result proven in the previous induction problem, every finite nonempty set of integers contains a maximal element. Let max(S) denote this maximum.
				\item max(S) is the least upper bound of S
				
				max(S) is an upper bound because $s \le max(S)$ for all $s \in S$. It is the least upper bound because no smaller natural number than max(S) can be an upper bound for S.
				\item Since every nonempty bounded-above subset $S \subseteq N$ has a least upper bound $max(S) \in N$. N is complete.
			\end{itemize}
		\subsection*{Exercise 1.27}
			For each item, compute the requested supremum or infimum or carefully explain why it does not exist. Either way, prove that your answer is correct.
			\begin{itemize}
				\item Determine sup(A) for A = $\{\frac{(-1)^n}{n}:n \in N\}$
				
					The set A alternates between positive and negative terms: A = \{-1, 1/2, -1/3, 1/4, -1/5 ...\}. The positive terms are 1/2, 1/4, 1/6 ... approaching 0; the negative terms are -1, -1/3, -1/5 ... approaching 0. The supremum is 1/2. 
					
					Proof
					\begin{itemize}
						\item For all n, $\frac{(-1)^n}{n} \le 1/2$
						\item if $n \ge 2$, positive terms $1/n \le 1/2$ and negative terms are < 1/2
						\item Thus, $sup(A) = 1/2$ and $1/2 \in A$
					\end{itemize}
				
				\item Fix $a \in (0,1)$. Determine inf(B) for $B = \{a^n:n \in N\}$.
				
				For $a \in (0,1)$, $a^n$ is a strictly decreasing sequence bounded below by 0. For example, with a = 0.5: $B = \{0.5, 0.25, 0.125\}$
				
				To prove inf(B) = 0, firstly, $a^n > 0$ for all n, so 0 is a lower bound. For any $\epsilon > 0$, $choose N > \frac{ln \epsilon}{ln \alpha}$. Then $a^N \le \epsilon$, proving 0 is the greatest lower bound.
				
				\item Fix $a \in (1, \infty)$. Determine sup(C) for $C = \{a^n: n \in N\}$
				
				For $a > 1, a^n$ is strictly increasing sequence unbounded above. For example a = 2. Thus the supremum doesn't exist in the real numbers. Since for any M > 0, choose N > $\frac{ln M}{ln a}$. Then $a^N > M$, proving the sequence grows without bound.
			\end{itemize}
		\subsection*{Exercise 1.28}
			Prove the infimum case of Theorem 1.24
			
			To prove the infimum case of theorem 1.24, we demonstrate that for a set $A \subseteq R$, $inf(A) = \beta$ if and only if 1. $\beta$ is a lower bound of A 2. For every $\epsilon > 0$, there exists $x \in A$ such that $x < \beta + \epsilon$
			\begin{itemize}
				\item forward direction: assume $\beta = inf(A)$
				\begin{itemize}
					\item Lower bound property: by definition of infimum, $\beta$ is a lower bound of A, so $x \le \beta$ for all $x \in A$.
					\item $\epsilon$ condition: Let $\epsilon > 0$. Since $\beta$ is the greatest lower bound, $\beta + \epsilon$ is not a lower bound of A. Thus, there exists $x \in A$ such that $x < \beta + \epsilon$
				\end{itemize}
				\item reverse direction: assume $\beta$ is a lower bound of A and for every $\epsilon > 0$, there exists $x \in A$ with $x < \beta + \epsilon$. We prove $\beta = inf(A)$ by showing it is the greatest lower bound.
				\begin{itemize}
					\item $\beta$ is already a lower bound by assumption
					\item Suppose there exists a greater lower bound $\gamma$ such that $\gamma > \beta$. Let $\epsilon = \gamma - \beta > 0$. By assumption, there exists $x \in A$ such that $x < \beta + \epsilon = \gamma$. This contradicts $\gamma$ being a lower bound. Thus $\beta$ is the greatest lower bound.
				\end{itemize}
			\end{itemize}
		\subsection*{Exercise 1.29}
			Prove that $sup(\{\frac{n}{n+1}: n \in N\}) = 1$ and $inf(\{\frac{n}{n+1}: n \in N\}) = 1/2$.
			
			Sup(S) = 1
			\begin{itemize}
				\item show that 1 is an upper bound
				
				For all $n \in N$,
				\begin{equation}
					\frac{n}{n+1} = 1 - \frac{1}{n+1} < 1
				\end{equation}
				So every element of S is less than 1, hence 1 is an upper bound.
				\item Show that 1 is the least upper bound
				
				Let $\epsilon > 0$. We want to show there exists $x \in S$ such that $x > 1 - \epsilon$. Set $\frac{n}{n+1} > 1 - \epsilon$
				\begin{equation}
					1 - \frac{1}{n+1} > 1-\epsilon \rightarrow \frac{1}{n+1} < \epsilon \rightarrow n > 1/\epsilon - 1
				\end{equation}
				For any $\epsilon > 0$, pick n large enough so that $n > 1/\epsilon - 1$. Then $\frac{n}{n+1} > 1-\epsilon$.
				Thus, for any $\epsilon > 0$, there is an element of S within $\epsilon$ of 1 from below. Therefore, 1 is the supremum of S.
				
				\item inf(S) = 1/2
				\begin{itemize}
					\item Show that 1/2 is a lower bound. The smallest value occurs at $n = 1: 1/2$. For $n > 1$, $\frac{n}{n+1} \ge 1/2$
					\item Show that 1/2 is the greatest lower bound
					For any $\epsilon > 0$, $1/2+\epsilon$ is greater than 1/2. For n=1, $1/2 < 1/2+\epsilon$.
					For all n > 1, $\frac{n}{n+1} > 1/2$. Thus 1/2 is the smallest element and hence the infimum.
				\end{itemize}
			\end{itemize}
		\subsection*{Exercise 1.30}
			Let $A, B \subseteq R$, and assume that $sup(A) < sup(B)$
			\begin{itemize}
				\item Show that there exists an element $b \in B$ that is an upper bound for A.
				
					Since sup(A) < sup(B), sup(A) is not an upper bound for B. There exists an element $b \in B$ such that $b > sup(A)$. Since sup(A) is an upper bound for A, any $b > sup(A)$ is also an upper bound for A. Thus, $b \in B$ serves as an upper bound for A.
				\item Give an example to show that this is not necessarily the case if we instead only assume that $sup(A) \le sup(B)$. You do not need to prove your answer.
				
					Let $A = (0,1)$ and $B = (0,1).
					sup(A) = sup(B) = 1$. However, every element $b \in B$ satisfies $b < 1$, so no $b \in B$ is an upper bound for A.
			\end{itemize}
\end{document}